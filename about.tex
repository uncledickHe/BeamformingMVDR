\documentclass{article}
\usepackage[utf8]{inputenc}
\usepackage[cp1251]{inputenc}  %% 1
\usepackage[T2A]{fontenc}      %% 2
\usepackage[russian]{babel}    %% 3
\usepackage{amsmath}

\title{Формирование узкополосного луча MVDR.
Метод step.}
\author{Яцкин Данил}

\begin{document}

\maketitle

\section{Введение. Входные и выходные данные}
 
 
Для начала необходимо разобраться, что же такое MVDR. Сама аббревиатура расшифровывается как \textit{minimum variance distortionless response (неискаженный отклик с наименьшей дисперсией)}. Соответственно, в рамках MVDR рассматриваются методы и алгоритмы обработки информации (сигнала), способствующие удовлетворению заявленных выше требований.
Ключевым (и единственным) методом класса MVDR является метод step. 

Рассмотрим возможные входные и выходные параметры для этого метода.

\textbf{Входные параметры}

\begin{tabular}{c|c}
\hline
\textit{H} & объект \\
\hline
\textit{X} & входной сигнал \\
\hline
\textit{XT} & шаблоны для настройки \\
\hline
\theta & угол (направление)  \\
\hline


\end{tabular}\\

\textit{H} и \textit{X} являются обязательными для задания параметрами, \textit{XT} и \theta могут отдельно не определяться.

\textit{X} в общем случае представляется в виде матрицы, число строк в которой равно числу подрешеток антенной решетки (при наличии таковых). Число столбцов, соответственно, рано числу элементов в этих подрешетках. Если решетка не разбита на подрешетки, то  \textit{X} представляет из себя строку.

\textit{XT} представляется в аналогичном \textit{X} виде (см. выше).

\theta задается, как правило, в азимутальной системе координат.\\


\textbf{Выходные параметры}

\begin{tabular}{c|c}
\hline
\textit{Y} & преобразованный сигнал \\
\hline
\textit{W} & формирующие веса \\
\hline



\end{tabular}


И \textit{Y}, и \textit{W} представляются в виде матриц, число строк в которых равно числу подрешеток (при наличии таковых), а число столбцов - числу направлений формирования луча.


\section{Математические основы алгоритма (метода)}

Алгоритм MVDR способен качественно подавлять помехи, но, как правило, требуется высокое значение соотношения "сигнал-шум". При этом реализация алгоритма существенным образом зависит от векторов управляющих воздействий и, в том числе, от угла падения принимаемого сигнала по отношению к элементу антенны.

Соответственно, результирующий массив \textit{Y} выражается через исходный (\textit{X}) формулой 

\begin{equation}
Y = W^H X,
\end{equation}

где $W^H$ -- эрмитово-сопряженная мартица для W.

При известном направлении полезного сигнала следует минимизировать выходную мощность. Мощность выражается следующим образом:

%\begin{equation}
\begin{center}
$P = \{E|Y|^2\} = E\{W^H X X^H W\} = W^H E\{X X^H W\} = W^H R,$
\end{center}
%\end{equation}

где R - ковариационная матрица для X.\\ 

Соответственно, оптимальные веса выбираются именно для минимизации выходной мощности при сохранении единичного усиления в соответствующем направлении $a(\theta)$, что является вектором управления для рассматриваемого сигнала.

Адаптивный алгоритм MVDR можно записать следующим образом:

\begin{equation}
min_W \{W^H R W\} \text{  }\text{  } \text{  }\text{  }\text{          при условии } W^H a(\theta) = 1
\end{equation}

Управляющий вектор $a(\theta)$ задается следующим образом:

\begin{equation}
a(\theta) = \begin{pmatrix} 1 \\ exp(j \frac{2 \pi d}{\lambda} sin\theta_i) \\ exp(j(m-1) \frac{2 \pi d}{\lambda} sin\theta_i) \end{pmatrix},
\end{equation}

m -- число элементов.

Оптимизация весовых коэффициентов может быть записана  формулой:

\begin{equation}
W=\frac{R^{-1}a(\theta)}{a^H(\theta)R^{-1}a(\theta)}
\end{equation}


\end{document}
